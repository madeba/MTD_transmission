


%%% LaTeX Template: Two column article
%%%
%%% Source: http://www.howtotex.com/
%%% Feel free to distribute this template, but please keep to referal to http://www.howtotex.com/ here.
%%% Date: February 2011

%%% Preamble
\documentclass[	french,DIV=calc,%
							paper=a4,%
							fontsize=11pt,%
							twocolumn]{scrartcl}	 					% KOMA-article class

\usepackage{lipsum}													% Package to create dummy text
\usepackage[utf8x]{inputenc}
\usepackage[francais]{babel}	
% English language/hyphenation
\usepackage[protrusion=true,expansion=true]{microtype}				% Better typography
\usepackage{amsmath,amsfonts,amsthm}					% Math packages
\usepackage[pdftex]{graphicx}									% Enable pdflatex
\usepackage[svgnames]{xcolor}									% Enabling colors by their 'svgnames'
\usepackage[hang, small,labelfont=bf,up,textfont=it,up]{caption}	% Custom captions under/above floats
\usepackage{epstopdf}												% Converts .eps to .pdf
\usepackage{subfig}													% Subfigures
\usepackage{booktabs}												% Nicer tables
\usepackage{fix-cm}													% Custom fontsizes

\usepackage[squaren,Gray]{SIunits}

%%% Custom sectioning (sectsty package)
\usepackage{sectsty}													% Custom sectioning (see below)
\allsectionsfont{%															% Change font of al section commands
	\usefont{OT1}{phv}{b}{n}%										% bch-b-n: CharterBT-Bold font
	}

\sectionfont{%																% Change font of \section command
	\usefont{OT1}{phv}{b}{n}%										% bch-b-n: CharterBT-Bold font
	}



%%% Headers and footers
\usepackage{fancyhdr}												% Needed to define custom headers/footers
	\pagestyle{fancy}														% Enabling the custom headers/footers
\usepackage{lastpage}	

% Header (empty)
\lhead{}
\chead{}
\rhead{}
% Footer (you may change this to your own needs)
\lfoot{\footnotesize \texttt{Manip tomo} \textbullet Réglages optiques et logiciels}
\cfoot{}
\rfoot{\footnotesize page \thepage\ of \pageref{LastPage}}	% "Page 1 of 2"
\renewcommand{\headrulewidth}{0.0pt}
\renewcommand{\footrulewidth}{0.4pt}

\newcommand{\code}[1]{\texttt{#1}}


%%% Creating an initial of the very first character of the content
\usepackage{lettrine}
\newcommand{\initial}[1]{%
     \lettrine[lines=3,lhang=0.3,nindent=0em]{
     				\color{DarkGoldenrod}
     				{\textsf{#1}}}{}}



%%% Title, author and date metadata
\usepackage{titling}															% For custom titles

\newcommand{\HorRule}{\color{DarkGoldenrod}%			% Creating a horizontal rule
									  	\rule{\linewidth}{1pt}%
										}

\pretitle{\vspace{-30pt} \begin{flushleft} \HorRule 
				\fontsize{50}{50} \usefont{OT1}{phv}{b}{n} \color{DarkRed} \selectfont 
				}
\title{Documentation Tomo}					% Title of your article goes here
\posttitle{\par\end{flushleft}\vskip 0.5em}

\preauthor{\begin{flushleft}
					\large \lineskip 0.5em \usefont{OT1}{phv}{b}{sl} \color{DarkRed}}
\author{Matthieu Debailleul, }											% Author name goes here
\postauthor{\footnotesize \usefont{OT1}{phv}{m}{sl} \color{Black} 
					Laboratoire MIPS 								% Institution of author
					\par\end{flushleft}\HorRule}

\date{}																				% No date



%%% Begin document
\begin{document}
\maketitle
\thispagestyle{fancy} 			% Enabling the custom headers/footers for the first page 
% The first character should be within \initial{}
%\initial{D}\textbf{escription succincte des réglages optiques et des logiciels utilisés.}


\section{Principes}

L'interface graphique écrit dans 2 fichiers de configuration : \code{recon.txt}, \code{config\_manip.txt}. Les chemins utiles au démarrage se trouvent dans \$HOME/.config/gui\_tomo.config.


Pour enregistrer une modif dans un fichier sur le disque dur, il faut en fait réenregistrer toutes les données.
On utilise pour cela des tableaux de stockage, contiennent l'ensemble des données du fichier.
Ces tableaux stockés en mémoire sont modifiés à chaque clic sur un bouton.

À l'acquisition, les fichiers contenus dans \code{Projet\_tomo/tomo\_config} sont copiés dans le répertoire des données acquises (par défaut \code{/ramdisk/ACQUIS/}), afin de conserver les paramètres expérimentaux. 

\section{Pc d'acquisition}
\begin{enumerate}
 \item Le fichier de \code{config\_manip.txt} doit être modifié uniquement sur le PC d'acquisition. C'est pourquoi il existe un 2e chemin, spécifique au Pc d'acquisition, indiqué par \code{config\_manip\_pc\_acquis}. 
Sur un Pc de reconstruction uniquement, ce fichier \code{config\_manip.txt} ne doit jamais être modifié (c'est un fichier de sauvegarde qui informe sur les conditions d'acquisition).

\item L'endroit de sauvegarde du fichier  \code{recon.txt} n'est pas critique, et il est enregistré dans le répertoire des données. 
\item 
Le fichier \code{gui\_tomo.conf} est toujours sauvegardé dans \code{\$HOME/.config/} quelle que soit la machine. 
\end{enumerate}



\end{document}
