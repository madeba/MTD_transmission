


%%% LaTeX Template: Two column article
%%%
%%% Source: http://www.howtotex.com/
%%% Feel free to distribute this template, but please keep to referal to http://www.howtotex.com/ here.
%%% Date: February 2011

%%% Preamble
\documentclass[	french,DIV=calc,paper=a4,fontsize=10pt,
							twocolumn]{scrartcl}	 					% KOMA-article class
\usepackage{lmodern}
\usepackage{lipsum}													% Package to create dummy text
\usepackage[utf8x]{inputenc}
\usepackage[francais]{babel}	
% English language/hyphenation
\usepackage[protrusion=true,expansion=true]{microtype}				% Better typography
\usepackage{amsmath,amsfonts,amsthm}					% Math packages
\usepackage[pdftex]{graphicx}									% Enable pdflatex
\usepackage[svgnames]{xcolor}									% Enabling colors by their 'svgnames'
\usepackage[hang, small,labelfont=bf,up,textfont=it,up]{caption}	% Custom captions under/above floats
\usepackage{epstopdf}												% Converts .eps to .pdf
\usepackage{subfig}													% Subfigures
\usepackage{booktabs}												% Nicer tables
\usepackage{fix-cm}	
\usepackage{listings}% 
%\usepackage{bclogo} ne compile pas en pdflatex
\usepackage[tikz]{bclogo}
\usepackage[squaren,Gray]{SIunits}
\usepackage{adjustbox}%marge pour fcolorbox
%%% Custom sectioning (sectsty package)
\usepackage{sectsty}													% Custom sectioning (see below)
\allsectionsfont{%															% Change font of al section commands
	\usefont{OT1}{phv}{b}{n}%										% bch-b-n: CharterBT-Bold font
	}

\sectionfont{%																% Change font of \section command
	\usefont{OT1}{phv}{b}{n}%										% bch-b-n: CharterBT-Bold font
	}

\lstset{language=C++,
                basicstyle=\ttfamily,
                keywordstyle=\color{blue}\ttfamily,
                stringstyle=\color{red}\ttfamily,
                commentstyle=\color{green}\ttfamily,
                morecomment=[l][\color{magenta}]{\#}
}

%%% Headers and footers
\usepackage{fancyhdr}												% Needed to define custom headers/footers
	\pagestyle{fancy}														% Enabling the custom headers/footers
\usepackage{lastpage}	

% Header (empty)
\lhead{}
\chead{}
\rhead{}
% Footer (you may change this to your own needs)
\lfoot{\footnotesize \texttt{Calcul} \textbullet Bases}
\cfoot{}
\rfoot{\footnotesize page \thepage\ of \pageref{LastPage}}	% "Page 1 of 2"
\renewcommand{\headrulewidth}{0.0pt}
\renewcommand{\footrulewidth}{0.4pt}

\newcommand{\code}[1]{\texttt{#1}}


%%% Creating an initial of the very first character of the content
\usepackage{lettrine}
\newcommand{\initial}[1]{%
     \lettrine[lines=3,lhang=0.3,nindent=0em]{
     				\color{DarkGoldenrod}
     				{\textsf{#1}}}{}}

\newcommand{\boitegrise}[1]{\fcolorbox{lightgray}{SteelBlue}{\hspace{0.5mm} \textcolor{Cornsilk}{\textbf{{\tiny #1}}} }
}

%%% Title, author and date metadata
\usepackage{titling}															% For custom titles

\newcommand{\HorRule}{\color{DarkGoldenrod}%			% Creating a horizontal rule
									  	\rule{\linewidth}{1pt}%
										}

\pretitle{\vspace{-30pt} \begin{flushleft} \HorRule 
				\fontsize{30}{30} \usefont{OT1}{phv}{b}{n} \color{DarkRed} \selectfont 
				}
\title{Découpe Hors-axe}					% Title of your article goes here
\posttitle{\par\end{flushleft}\vskip 0.5em}

\preauthor{\begin{flushleft}
					\large \lineskip 0.5em \usefont{OT1}{phv}{b}{sl} \color{DarkRed}}
\author{}											% Author name goes here
\postauthor{\footnotesize \usefont{OT1}{phv}{m}{sl} \color{Black} 
													% Institution of author
					\par\end{flushleft}\HorRule}

\date{}																				% No date



%%% Begin document
\begin{document}
\maketitle
\thispagestyle{fancy} 			% Enabling the custom headers/footers for the first page 
% The first character should be within \initial{}


L'extraction du champ complexe depuis un hologramme hors-axe est simple : il suffit de découper le spectre de l'ordre +1 ou -1 aux coordonnées de l'onde porteuse, et aux dimensions du spectre fixées par l'ouverture du numérique de l'objectif et l'échantillonnage $\Delta f$.

L'hologramme étant une image réelle, son spectre présente une symétrie hermitienne : le spectre est impaire et l'objet et le jumeau son complexe conjugués.  La bibliothèque \code{fftw} 
permet d'exploiter cette symétrie afin de ne calculer qu'un demi-spectre (en réalité, un demi spectre +1 pixel selon l'axe $y$) avec la fonction de création de plan  :   \code{fftw\_plan\_dft\_r2c\_2d}.

Plusieurs fonctions ont donc été développées. Dans l'ordre croissant de vitesse  :
\begin{enumerate}
\item \code{holo2TF\_UBorn} Le spectre est calculé en c2c, resymétrisé et recentré avec \code{fftshift}.
\item \code{holo2TF\_UBorn2} Le spectre est calculé en r2c, resymétrisé et recentré avec \code{fftshift}. 
 \item \code{holo2TF\_UBorn2\_shift} : la fft est calculée en r2c, mais resymétrisée. Le spectre n'est pas redécalé apres la TF. ON découpe dans le spectre éclaté aux 4 coins, ce qui gagne 1 fftshift.
 \item \code{holo2TF\_UBorn2\_shift\_r2c} : la fft est calculée en r2c simple (pas de symétrisation). Le spectre n'est aps redécalé. On gagne donc sur la symétrisation et le fftshift. C'est la méthode la plus rapide.
\end{enumerate}

Les gain dépendent donc beaucoup des fft calculé. 
\begin{table}
 


\begin{center}
\begin{tabular}{|l|l|l|l|l|}\hline
             &c2c & r2c symétrisée & r2c & fftshift 1024\\\hline
1 appel (ms) &18,4 & 11,7 & 9,2 & 2,4\\\hline
600 appel (s)&11,0 & 7,0 & 5,5 & 1,44\\\hline
\end{tabular}
\caption{Temps pour les différentes fft en 1024x1024, 64 bits, outplace, FFTW\_MEASURE, 3 threads sur un core i5-3550.}
\end{center}
\end{table}

\end{document}
