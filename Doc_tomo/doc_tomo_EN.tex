


%%% LaTeX Template: Two column article
%%%
%%% Source: http://www.howtotex.com/
%%% Feel free to distribute this template, but please keep to referal to http://www.howtotex.com/ here.
%%% Date: February 2011

%%% Preamble
\documentclass[	french,DIV=calc,%
							paper=a4,%
							fontsize=11pt,%
							twocolumn]{scrartcl}	 					% KOMA-article class

\usepackage{lipsum}													% Package to create dummy text
\usepackage[utf8x]{inputenc}
\usepackage[francais]{babel}	
% English language/hyphenation
\usepackage[protrusion=true,expansion=true]{microtype}				% Better typography
\usepackage{amsmath,amsfonts,amsthm}					% Math packages
\usepackage[pdftex]{graphicx}									% Enable pdflatex
\usepackage[svgnames]{xcolor}									% Enabling colors by their 'svgnames'
\usepackage[hang, small,labelfont=bf,up,textfont=it,up]{caption}	% Custom captions under/above floats
\usepackage{epstopdf}												% Converts .eps to .pdf
\usepackage{subfig}													% Subfigures
\usepackage{booktabs}												% Nicer tables
\usepackage{fix-cm}													% Custom fontsizes

\usepackage[squaren,Gray]{SIunits}

%%% Custom sectioning (sectsty package)
\usepackage{sectsty}													% Custom sectioning (see below)
\allsectionsfont{%															% Change font of al section commands
	\usefont{OT1}{phv}{b}{n}%										% bch-b-n: CharterBT-Bold font
	}

\sectionfont{%																% Change font of \section command
	\usefont{OT1}{phv}{b}{n}%										% bch-b-n: CharterBT-Bold font
	}



%%% Headers and footers
\usepackage{fancyhdr}												% Needed to define custom headers/footers
	\pagestyle{fancy}														% Enabling the custom headers/footers
\usepackage{lastpage}	

% Header (empty)
\lhead{}
\chead{}
\rhead{}
% Footer (you may change this to your own needs)
\lfoot{\footnotesize \texttt{Manip tomo} \textbullet Réglages optiques et logiciels}
\cfoot{}
\rfoot{\footnotesize page \thepage\ of \pageref{LastPage}}	% "Page 1 of 2"
\renewcommand{\headrulewidth}{0.0pt}
\renewcommand{\footrulewidth}{0.4pt}

\newcommand{\code}[1]{\texttt{#1}}


%%% Creating an initial of the very first character of the content
\usepackage{lettrine}
\newcommand{\initial}[1]{%
     \lettrine[lines=3,lhang=0.3,nindent=0em]{
     				\color{DarkGoldenrod}
     				{\textsf{#1}}}{}}



%%% Title, author and date metadata
\usepackage{titling}															% For custom titles

\newcommand{\HorRule}{\color{DarkGoldenrod}%			% Creating a horizontal rule
									  	\rule{\linewidth}{1pt}%
										}

\pretitle{\vspace{-30pt} \begin{flushleft} \HorRule 
				\fontsize{50}{50} \usefont{OT1}{phv}{b}{n} \color{DarkRed} \selectfont 
				}
\title{Documentation Tomo}					% Title of your article goes here
\posttitle{\par\end{flushleft}\vskip 0.5em}

\preauthor{\begin{flushleft}
					\large \lineskip 0.5em \usefont{OT1}{phv}{b}{sl} \color{DarkRed}}
\author{Matthieu Debailleul, }											% Author name goes here
\postauthor{\footnotesize \usefont{OT1}{phv}{m}{sl} \color{Black} 
					Laboratoire MIPS 								% Institution of author
					\par\end{flushleft}\HorRule}

\date{}																				% No date



%%% Begin document
\begin{document}
\maketitle
\thispagestyle{fancy} 			% Enabling the custom headers/footers for the first page 
% The first character should be within \initial{}
\initial{B}\textbf{rief description of the optical set-up and the useful softwares.}


\section{Useful programms}

\subsection*{Show\_fourier : réglages du hors-axe et balayage angulaire}

Les deux dernières étapes sont faites avec le programme \textbf{showfourier}, qui affiche la TF de l'objet.  en indiquant à quel 
endroit doit se situer le spectre objet cohérent (ordre +1).



\subsection{Compilation :  bibliothèques externes}
Les bibliothèques (hors Debian) utiles aux programmes ont été placées dans 
\texttt{/opt/tomo\_lib/} et la variable d'environnement \texttt{\$LD\_LIBRARY\_PATH} réglé en conséquence (\texttt{/etc/ld.conf}, \texttt{.bashrc}). Les sources des programmes sont disponibles 
dans \texttt{Tomo\_prog}, et les binaires ont été placés dans le 
\texttt{/usr/bin/}. 

\subsection{Acquisition : manip\_tomo}
Elle se lance avec la commande manipTomo5 : 
    \begin{center}
    \texttt{manipTomo5 -ni 600 -voffset 0 -0.1 \\-vfleur 4.2 3}   
    \end{center}
    
 pour 600 images acquises. 
Les paramètres \texttt{voffset} et \texttt{vfleur} dépendant du réglage du balayage (décalage du zéro et intervalle max de tension  $v_x$ et $v_y$, réglées à l'aide de \code{show\_fourier}).

La nouvelle version ne prend aucun paramètre en option : ils sont lus dans les fichiers recon.txt et config.txt.  
On lance simplement  l'acquisition avec \textbf{\code{manip\_tomo5}}

\subsection{Reconstruction}
\subsubsection{Exécutables}
Elle se fait en 2 étapes : prétraitement puis reconstruction à proprement parler. 

Le prétraitement  extrait les champs complexes  des hologrammes (hors-axe, aberration, déroulement \ldots). La commande nécessite le chemin vers les acquisitions 
Les coordonnées du centre du cercle de hors-axe (lui-même réglé avec le programme \texttt{show\_fourier}) sont 

    \begin{center}
    \texttt{Tomo\_pretraitement -i /ramdisk/ACQUIS}
    \end{center}

Les champs complexes sont enfin utilisés pour la reconstruction.


La dernière version ne prend
plus de paramètres en options : ils sont lus dans les fichiers de configuration et se lance avec \code{Tomo\_pretraitement}.


\subsubsection{Configuration}
Le pretraitement et la reconstruction sont contrôlés par deux fichiers situés dans le repertoire \code{Tomo\_config}~:
\begin{enumerate}
 \item config\_manip.txt
 \item recon.txt
\end{enumerate}

Le fichier de configuration de la manip est a priori invariant une fois la manip fixée (longueur d'onde, grandissement total, NA etc.).
Le fichier recon.txt permet de contrôler le pretraitement et la reconstruction (cf \textsc{Tab.} \ref{recon_tab}). 

\textbf{
L'ensemble des chemins (recon.txt, config\_manip.txt, acquisitions) doit être indiqué dans \$HOME/.config./gui\_tomo.conf}


\begin{table}[h!]
\caption{Mot-clés contrôlant la reconstruction}
\centering
	\begin{tabular}{ll}
		\toprule
		\multicolumn{2}{c}{\code{Tomo\_config/recon.txt}} \\
		\cmidrule(r){1-2}
			Mot-clé 	& Fonction  \\
		\midrule
			BORN 		&  Approx. utilisée (0=RYTOV) \\
			C\_ABER  	& 1=Corriger les aberrations  \\
			DEROUL		& 1=Dérouler la phase\\
			NB\_HOLO\_RECON	& Nbre d'hologrammes à traiter\\
		\bottomrule
	\end{tabular}\label{recon_tab}
\end{table}

L'approximation de  Rytov est meilleure pour les objets épais ($>5\micro$m), mais nécessite un déroulement de phase.


\subsection{Correction des aberrations}

Les aberrations résiduelles peuvent être  corrigées (\texttt{C\_ABER=1} dans \texttt{recon.txt}) en analysant le fond des acquisitions, dont la phase est supposée plane. L'écart à la planéité fournit le polynôme de
correction des aberrations. 

Pour fonctionner de façon optimale, la correction d'aberration nécessite un masque binaire séparant l'objet du fond, fourni par l'utilisateur (en l'absence de masque, une masque unité
est créé).


\section{Réglages de la manip}
Le réglage  optique du tomo demande d'effectuer différentes étapes~:

\begin{enumerate}
 \item planéité de l'onde d'illumination sur l'objet
 \item accord de phase entre référence et illumination sur la caméra
 \item conjugaison des diaphragmes de champ avec l'objet et la caméra
 \item réglage du hors-axe
 \item réglage du balayage
\end{enumerate}


\subsection{eBusPLayer : utilisation et réglages caméra}
Voltages\_BlackLevelOffset (eBusPlayer 4.1.5) : la valeur 101 permet d'avoir à la fois un zéro (capteur couvert) et le niveau de saturation.  
Ce paramètre disparait avec la nouvelle caméra (MV1-D2048-96-G2-10, pixel de 5,5 µm).

\subsubsection{Réglages réseaux}
La caméra est sur la 3è carte ethernet (pci) déclarée en \texttt{eth2}. Les jumbo frames (trames géantes) doivent être activée avec 
l'option \texttt{MTU=9000}, sinon 
la caméra plafonne à 80 IPS au lieu de 90. La carte ethernet prends un adresse locale de type \texttt{169.168.0.2}. 
Il faut enfin fixer une adresse IP pour la caméra. 

\section{Bruit cohérent}
Le bruit cohérent peut être corrigé en réalisant une acquisition à vide, qui peut être soustraite à l'image finale.

\section{Interface graphique}

Le fichier de configuration des chemins utiles se trouve dans \code{\$HOME/.config/gui\_tomo.conf}. Il fournit à l'interface graphique un fichier L'interface graphique enregistre les paramètres dans les fichiers de 
configurations \code{recon.txt} et \code{config\_manip.txt}, puis lance les binaires d'acquisition, de pretraitement et reconstruction.
\end{document}
